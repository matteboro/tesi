\chapter*{Introduzione} %Se si cambia il Titolo cambiare anche la riga successiva così che appia corretto nell'conclusione
\addcontentsline{toc}{chapter}{Introduzione} %Per far apparire Introduzione nell'indice (Il nome deve rispecchiare quello del chapter)

Nel capitolo \hyperref[chapter:background]{Background} verranno esposte le basi teoriche e matematiche per comprendere il concetto di interpretazione astratta e di come questa può essere utilizzata per analizzare staticamente i programmi. In particolare cercheremo di capire la suddivisione in fasi dell'analisi tramite interpretazione astratta. In più daremo un metodo costruttivo per derivare un dominio più preciso da un dominio astratto già definito. Questi argomenti ci introduranno al secondo capitolo, \hyperref[chapter:decoupling]{Decoupling della fase ascendente e discendente}, dove verrà esposta dal punto di vista teorico l'idea proposta dall'articolo \cite{DBLP:conf/aplas/ArceriMZ22} per migliorare la precisione dell'analisi tramite interpretazione astratta senza pagare troppo in termini di efficienza. Per concludere, nell'ultimo capitolo, \hyperref[chapter:lisa]{Implementazioni in LiSA}, verranno mostrate come questi concetti sono stati applicati all'interno della libreria LiSA che utilizza l'interpretazione astratta per fare analisi statica di programmi. Si partirà da una descrizione generale della architettura di LiSA per poi passare a una descrizione dettagliata delle implementazioni della fase ascendente, del dominio dei sottoinsiemi non ridondanti di un dominio ed infine del decoupling delle fasi.