\chapter{Conclusione} %Se si cambia il Titolo cambiare anche la riga successiva così che appia corretto nell'conclusione
\addcontentsline{toc}{chapter}{Conclusione} %Per far apparire Introduzione nell'indice (Il nome deve rispecchiare quello del chapter)
Lo scopo della tesi è stato quello di introdurre in LiSA l'idea proposta nell'Articolo \cite{DBLP:conf/aplas/ArceriMZ22} e esposta nel capitolo sul \hyperref[chapter:decoupling]{decoupling}, ovvero di permettere all'utente di LiSA un modo semplice e diretto per utilizzare domini astratti diversi per le due fasi dell'analisi. Nel Capitolo \hyperref[chapter:decoupling]{Decoupling delle fase ascendente e discendente} abbiamo anche visto come il decoupling sia utile solo se c'è un incremento di precisione nel dominio  astratto usato nella fase discendente rispetto a quello usato nella fase ascendente. In relazione a questo concetto quindi è stato esposto nel Capitolo \hyperref[chapter:background]{Background} alla Sezione \ref{sec:EgliMilnerWidening} un dominio astratto costruibile a partire da un dominio astratto già definito e inerentemente più preciso di quest'ultimo, ovvero il dominio dei sottoinsiemi finiti e non ridondanti di un reticolo. Le implementazioni in LiSA, esposte nel Capitolo \ref{chapter:lisa}, si sono concentrate in tre diverse fasi: (i) per prima cosa è stata aggiunta la possibilità di calcolare anche la fase discendente nell'algoritmo della ricerca del fix-point di LiSA, (ii) sono state aggiunte classi generiche base che semplificano la creazione dei domini dei sottoinsiemi non ridondanti a partire da domini già presenti in LiSA (come \texttt{Interval} e \texttt{Sign}) ed infine (iii) è stata aggiunta una rudimentale e non definitiva versione del decoupling che permette di utilizzare domini diversi nelle due fasi dell'interpretazione astratta.

\section{Lavori futuri}
Gli sviluppi di questo lavoro sono sia teorici che pratici. Per questi ultimi parliamo di LiSA, dove può essere innanzittuo inserito il decoupling delle fasi in modo definitivo e stabile. Anche lo sviluppo del dominio dei sottoinsiemi non ridondanti può essere espanso e migliorato. Per prima cosa si potrebbe avere un metodo totalmente automatico per la creazioni di questi. Tutte le informazioni necessarie infatti si trovano all'interno dei domini sottostanti. Come abbiamo visto però per i domini non-relazionali questo non è possibile, è infatti necessario ridefinire i metodi legati ad \texttt{assume}, anche se in un certo senso le informazioni necessarie alla loro definizione sono già all'interno del dominio sottostante, ma ricavarle risulta macchinoso e poco efficiente, se non impossibile. Al costo di riorganizzare queste classi si potrebbe quindi creare un metodo automatico e diretto per la creazione dei domini dei sottinsiemi non ridondanti e possibilmente, accoppiato al decoupling, una configurazione che permette di dire a LiSA di usare il dominio base nella fase ascendente e quello dei suoi sottinsiemi non ridondanti nella fase discendente, senza che all'utente sia richiesto di definire classi in più oltre quella base (anche l'oggetto \texttt{Decoupler} visto nella Sezione \ref{sec:lisaDecoupling} di fatto avrebbe un comportamento generico e comune in tutti questi casi). 

Per quanto riguarda gli sviluppi teorici del lavoro sono esposti nel dettaglio nella conclusione dell'articolo \cite{DBLP:conf/aplas/ArceriMZ22}. Viene spiegato che il decoupling delle fasi è un caso specifico di un concetto più ampio, denominato da Cousot, Giacobazzi e Ranzato \textit{meta-interpretazione astratta} o anche \textit{Abstract\(^2\) Interpretation} (\(A^2I\)) nell'articolo \cite{10.1145/3290355}. 